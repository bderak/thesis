\documentclass[english]{tktltiki}
\usepackage[pdftex]{graphicx}
\usepackage{subfigure}
\usepackage{url}
\begin{document}
%\doublespacing
%\singlespacing
\onehalfspacing

\title{Maximizing Influence in Big Social Networks}
\author{Behrouz Derakhshan}
\date{\today}

\maketitle


\section{Topic Specification}
Influence maximization in social networks is a topic that has attracted many researchers for more than a decade. People always influence each other in context of groups and social networks. The influence can be in form of life style choices, daily habits, purchasing habits among other things. A study by \cite{christ07} showed that obese people can increase the chance of of obesity among their peer. This topic also has an application in marketing, where by identifying influential people in a social network, they can increase the benefit through effective marketing campaign and relying on the spread of information about the product throughout the network. The key problem here first tackled by \cite{domingo01} is to find the set of influential people. 
\cite{kempe03} proposed two propagation model and have approach the problem this way; given a graph of nodes with arc probabilities find an initial set of size k that maximizes the expected size of propagation. Hence the problem is now to solve a optimization problem of finding the initial set, although NP hard but it can be solved using greedy methods in a more efficient manner. In \cite{kempe03} and other similar methods the arc probabilities/influences were always calculated with some trivial models. For example, two widely used methods were uniform link probabilities (all links having the same probability) or another method where link probabilities are selected uniformly at random from the set {0.1, 0.01, 0.001}. Using the provided link probabilities and the propagation models, experiments are run on the graph to find the initial set of users which yields the maximum benefit.  \\
A more advance method therefore has been studied, in which the arc probabilities are learned through mining the past propagation. \cite{saito08} were the first to study how to learn probabilities from past propagation by transforming the problem into a likelihood maximization and applying Expectation Maximization algorithm to solve it. A more direct mining approach has been proposed by \cite{goyal11} which eliminates the need for running simulations on the graph to find the optimal initial set of users. This method, which is the one of the central drivers for the thesis, directly computes the probability of a node being reached from some initial set of users(targeted users) . \\
With the help of online social networks, extremely big past propagation logs and social graphs can be found, and running the proposed algorithms in the papers mentioned above in their original form on these data sets will suffer scalability issues. Over past years, big data technologies have received a great deal of attention, and currently many frameworks and methodologies are available for doing computation on extremely large data sets. Hadoop (MapReduce) is one such framework, writing implementation of algorithms proposed by other researchers or a completely new algorithm in MapReduce will be another focus of this Thesis project. 

\section{Challenges}
Complexity of the task itself, especially when dealing with large amount of data is one main the challenges that I am trying to solve. 
According to \cite{Bon11} here are some other open areas that worth investigating :
\begin{itemize}
\item Learning the strength of influence exerted from a user
\item Temporal dimension (role of time)
\item Market competition (role that competing products will play in influence maximization for marketing)
\item General tendency of user characteristics for buying certain products and using those information to better find optimal target users (for example teenagers are more likely to buy video games than seniors)
\end{itemize}

\section{Resources}
Although the thesis work is not going to be part of my job description at the company that I work, Rovio Entertainment has a database of users which I might be able to access for performing experiments on real data sets. {check with Ari}.
Beside from the papers referenced in the previous sections, some other publications such as \cite{domingo02}, \cite{goyal10}, and \cite{cheng13} are among the related resources to the topic .\\
\pagebreak
\bibliographystyle{tktl}

\bibliography{specbib}
\lastpage

\pagestyle{empty}


\end{document}


