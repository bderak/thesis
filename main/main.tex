\documentclass[english]{tktltiki}
\usepackage[pdftex]{graphicx}
\usepackage{subfigure}
\usepackage{url}
\begin{document}
%\doublespacing
%\singlespacing
\onehalfspacing

\title{Big Data Influence Maximization blah balh}
\author{Behrouz Derakhshan}
\date{\today}

\maketitle

\numberofpagesinformation{\numberofpages\ pages + \numberofappendixpages\ appendices}
\classification{\protect{\ \\
A.1 [Introductory and Survey],\\
I.7.m [Document and text processing]}}

\keywords{layout, summary, list of references}

\begin{abstract}

Network influence maximization. Something about graphs, about network influcence, social network, other related areas, big data, hadoop and other technologies, graph processing

\end{abstract}

\mytableofcontents




\section{Introduction}
What are graphs, social graphs. Influence maximization in social networks. Its application in marketing.  \cite{kempe03} has defined models in which network propagation can be studied. 
\cite{domingo01} studied how to find the value of a customer in a network and how to use that to influence other customers.




\section{Literature Review}



\subsection{Graph}
%\enlargethispage{5mm}




\subsection{Social Graphs}




\subsection{Influence Maximization}
In \cite{kempe03} they have studied models by which influence propagates through social networks. They have discussed two diffusion models \\
Linear Threshold Model (add reference) and Independent Cascade Models(add reference) . Describe the two models briefly. 
The problem of influence maximization which is finding an initial set that will maximizes the profit( target function) is NP-hard . 
Formal definition : it asks for a parameter k, to find a k-nod set of maximum influence. They used approximation with a natural greedy hill-climbing strategy which will guarantee a 63 percent of the optimal solution performance (reference) . For the approximation to work , he target function should be sub-modular, they have proved that for both models the target function is indeed sub-modular.  They have run experiments on co-authorships in physics theory section of arXiv (www.arxiv.org) . The results show that their greedy algorithm performs better that other methods such as random selection, high-degree or central nodes discussed in book Social Network Analysis by S. Wasserman (find reference) \\
\cite{domingo01} discusses the use case of influence maximization in marketing. They first, described how to model a market as a social network. Making each individual user only dependent on its neighbours, the product and the marketing action(Markov random field). Through these a function was devised that calculates the global lift in profit given a marketing action was applied to some users. Starting from an initial configuration, using different optimization algorithms a local maxima for the function can be reached. 
They experimented on a collaborative filtering dataset (MovieLenz) in which users have rated a list of movies. They have first constructed the network using the user rating, by choosing similar users( calculating the Pearson correlation coefficient of users) as neighbours . They have performed experiments on the data set and compared their method with mass marketing and direct marketing. Their method has increased the profit the most. Due to non-linearly of the model, it did not scale well for bigger size networks, hence they have proposed a new linear model \cite{domingo02}. Not only the new model decreased the computation time time, the simplified equation for calculating the network value of customers, made it easier to incorporate more complex marketing action. They have experimented with Epinions (cite) data set. Which is a website where users can rate items. It also has a feature called trusted users, where each user can select many other users as trusted sources of reviews. Domingo et .al, hence, applied their model to the dataset, but considering trusted users of a user his neighbours. Their experiments showed that a continuous marketing action (a marketing action that can have unlimited values, such as a discount) performs much better than boolean marketing actions (one that can either be true or false, such whether or not a discount should be given to a user with no regards for the amount of the discount ). 







\subsection{Big Data and hadoop}


\subsection{Use cases reasearch and industry}



\section{Big data and Network influence maximization}

Description of what I'm going to do, big data/hadoop/mapreduce implementions for network influence


\subsection{Tools}




\subsection{Data set}





\subsection{Algorithm}
\subsubsection{Algorithm that I'm going to user}
\subsubsection{Big data implementaiton of the Algorithm}



\subsection{Other challenges and open questions}



\section{Conclusion}
In conclusion
\pagebreak



\bibliographystyle{tktl}

\bibliography{bibliography}

\lastpage

\appendices

\pagestyle{empty}


\end{document}


