\documentclass[english]{tktltiki}
\usepackage[pdftex]{graphicx}
\usepackage{subfigure}
\usepackage{url}
\begin{document}
%\doublespacing
%\singlespacing
\onehalfspacing

\title{My Topic}
\author{Behrouz Derakhshan}
\date{\today}

\maketitle

\numberofpagesinformation{\numberofpages\ pages + \numberofappendixpages\ appendices}
\classification{\protect{\ \\
A.1 [Introductory and Survey],\\
I.7.m [Document and text processing]}}

\keywords{layout, summary, list of references}

\begin{abstract}

Network influence maximization. Something about graphs, about network influcence, social network, other related areas, big data, hadoop and other technologies, graph processing

\end{abstract}

\mytableofcontents




\section{Introduction}
In \cite{kempe03} they have studiedmodels by which influce propogates through social networks. They have discussed two diffusion models \\
Linear Threshold Model (add reference) and Independent Cascade Models(add reference) . Describe the two models briefly. 
The problem of influence maximization which is finding an initial set that will maximizes the profit( target function) is NP-hard . 
Formal definition : it asks for a parameter k, to find a k-nod set of maximum influence. They used approximation with a natural greedu hill-climbing strategy which will guarantee a 63 percent of the optimal solution performance (reference) . For the approximation to work te target function should be submodular, they have proved that for both models the target function is indeed submodular.  They have run experiments on co-authorships in physicas theory section of arXiv (www.arxiv.org) . The results show that their greedy algorithm performs better that other methods such as random selection, high-degree or central nodes discussed in book Social Network Analysis by S. Wasserman (find reference) 





\section{Literature Review}

Let us start by looking at the sections expected to be in a scientific text. Keep in mind that the same expectations go for all kinds of technical writing.

\subsection{Graph}
%\enlargethispage{5mm}




\subsection{Social Graphs}




\subsection{Influcence Maximization}

The nature of the matter at hand determines how the topic chapters are disposed.

In order to guide the reader, it is a good idea to start each main chapter with a short paragraph 
on what the main topic of the chapter is and how it progresses from one sub-chapter to the next.

Texts with only one sub-chapter, or with more than two chapter levels (main and sub-chapters) are a 
sign of a problem with the disposition of the text. There may be justifiable reasons to use three-level 
headings in some technical documents, but they are an exception to the rule.

\subsection{Big Data and hadoop}

%\pagebreak
\subsection{Use cases reasearch and industry}


So-called mnemonic references are used for referring to sources; they are constructed as described 
in the section on the list of references. The page numbers should be added to the reference if it would 
be too laborious for the reader to find the reference in the source without them. 

References are always placed inside sentences.  This means that e.g. a separate reference at the end of a paragraph would be inappropriate.

The structure of the text must clearly show what the reference relates to.  At the same time, it 
shows how long a piece of the text that the reference relates to.


\section{Big data and Network influence maximization}

TDescription of what I'm going to do, big data/hadoop/mapreduce implementions for network influence


\subsection{Tools}




\subsection{Data set}





\subsection{Algorithm}



\subsection{Other challenges}



\section{Conclusion}



\nocite{*}
\bibliographystyle{tktl}

\bibliography{bibliography}

\lastpage

\appendices

\pagestyle{empty}


\end{document}


